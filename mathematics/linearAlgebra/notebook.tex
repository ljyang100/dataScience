
% Default to the notebook output style

    


% Inherit from the specified cell style.




    
\documentclass[11pt]{article}

    
    
    \usepackage[T1]{fontenc}
    % Nicer default font (+ math font) than Computer Modern for most use cases
    \usepackage{mathpazo}

    % Basic figure setup, for now with no caption control since it's done
    % automatically by Pandoc (which extracts ![](path) syntax from Markdown).
    \usepackage{graphicx}
    % We will generate all images so they have a width \maxwidth. This means
    % that they will get their normal width if they fit onto the page, but
    % are scaled down if they would overflow the margins.
    \makeatletter
    \def\maxwidth{\ifdim\Gin@nat@width>\linewidth\linewidth
    \else\Gin@nat@width\fi}
    \makeatother
    \let\Oldincludegraphics\includegraphics
    % Set max figure width to be 80% of text width, for now hardcoded.
    \renewcommand{\includegraphics}[1]{\Oldincludegraphics[width=.8\maxwidth]{#1}}
    % Ensure that by default, figures have no caption (until we provide a
    % proper Figure object with a Caption API and a way to capture that
    % in the conversion process - todo).
    \usepackage{caption}
    \DeclareCaptionLabelFormat{nolabel}{}
    \captionsetup{labelformat=nolabel}

    \usepackage{adjustbox} % Used to constrain images to a maximum size 
    \usepackage{xcolor} % Allow colors to be defined
    \usepackage{enumerate} % Needed for markdown enumerations to work
    \usepackage{geometry} % Used to adjust the document margins
    \usepackage{amsmath} % Equations
    \usepackage{amssymb} % Equations
    \usepackage{textcomp} % defines textquotesingle
    % Hack from http://tex.stackexchange.com/a/47451/13684:
    \AtBeginDocument{%
        \def\PYZsq{\textquotesingle}% Upright quotes in Pygmentized code
    }
    \usepackage{upquote} % Upright quotes for verbatim code
    \usepackage{eurosym} % defines \euro
    \usepackage[mathletters]{ucs} % Extended unicode (utf-8) support
    \usepackage[utf8x]{inputenc} % Allow utf-8 characters in the tex document
    \usepackage{fancyvrb} % verbatim replacement that allows latex
    \usepackage{grffile} % extends the file name processing of package graphics 
                         % to support a larger range 
    % The hyperref package gives us a pdf with properly built
    % internal navigation ('pdf bookmarks' for the table of contents,
    % internal cross-reference links, web links for URLs, etc.)
    \usepackage{hyperref}
    \usepackage{longtable} % longtable support required by pandoc >1.10
    \usepackage{booktabs}  % table support for pandoc > 1.12.2
    \usepackage[inline]{enumitem} % IRkernel/repr support (it uses the enumerate* environment)
    \usepackage[normalem]{ulem} % ulem is needed to support strikethroughs (\sout)
                                % normalem makes italics be italics, not underlines
    

    
    
    % Colors for the hyperref package
    \definecolor{urlcolor}{rgb}{0,.145,.698}
    \definecolor{linkcolor}{rgb}{.71,0.21,0.01}
    \definecolor{citecolor}{rgb}{.12,.54,.11}

    % ANSI colors
    \definecolor{ansi-black}{HTML}{3E424D}
    \definecolor{ansi-black-intense}{HTML}{282C36}
    \definecolor{ansi-red}{HTML}{E75C58}
    \definecolor{ansi-red-intense}{HTML}{B22B31}
    \definecolor{ansi-green}{HTML}{00A250}
    \definecolor{ansi-green-intense}{HTML}{007427}
    \definecolor{ansi-yellow}{HTML}{DDB62B}
    \definecolor{ansi-yellow-intense}{HTML}{B27D12}
    \definecolor{ansi-blue}{HTML}{208FFB}
    \definecolor{ansi-blue-intense}{HTML}{0065CA}
    \definecolor{ansi-magenta}{HTML}{D160C4}
    \definecolor{ansi-magenta-intense}{HTML}{A03196}
    \definecolor{ansi-cyan}{HTML}{60C6C8}
    \definecolor{ansi-cyan-intense}{HTML}{258F8F}
    \definecolor{ansi-white}{HTML}{C5C1B4}
    \definecolor{ansi-white-intense}{HTML}{A1A6B2}

    % commands and environments needed by pandoc snippets
    % extracted from the output of `pandoc -s`
    \providecommand{\tightlist}{%
      \setlength{\itemsep}{0pt}\setlength{\parskip}{0pt}}
    \DefineVerbatimEnvironment{Highlighting}{Verbatim}{commandchars=\\\{\}}
    % Add ',fontsize=\small' for more characters per line
    \newenvironment{Shaded}{}{}
    \newcommand{\KeywordTok}[1]{\textcolor[rgb]{0.00,0.44,0.13}{\textbf{{#1}}}}
    \newcommand{\DataTypeTok}[1]{\textcolor[rgb]{0.56,0.13,0.00}{{#1}}}
    \newcommand{\DecValTok}[1]{\textcolor[rgb]{0.25,0.63,0.44}{{#1}}}
    \newcommand{\BaseNTok}[1]{\textcolor[rgb]{0.25,0.63,0.44}{{#1}}}
    \newcommand{\FloatTok}[1]{\textcolor[rgb]{0.25,0.63,0.44}{{#1}}}
    \newcommand{\CharTok}[1]{\textcolor[rgb]{0.25,0.44,0.63}{{#1}}}
    \newcommand{\StringTok}[1]{\textcolor[rgb]{0.25,0.44,0.63}{{#1}}}
    \newcommand{\CommentTok}[1]{\textcolor[rgb]{0.38,0.63,0.69}{\textit{{#1}}}}
    \newcommand{\OtherTok}[1]{\textcolor[rgb]{0.00,0.44,0.13}{{#1}}}
    \newcommand{\AlertTok}[1]{\textcolor[rgb]{1.00,0.00,0.00}{\textbf{{#1}}}}
    \newcommand{\FunctionTok}[1]{\textcolor[rgb]{0.02,0.16,0.49}{{#1}}}
    \newcommand{\RegionMarkerTok}[1]{{#1}}
    \newcommand{\ErrorTok}[1]{\textcolor[rgb]{1.00,0.00,0.00}{\textbf{{#1}}}}
    \newcommand{\NormalTok}[1]{{#1}}
    
    % Additional commands for more recent versions of Pandoc
    \newcommand{\ConstantTok}[1]{\textcolor[rgb]{0.53,0.00,0.00}{{#1}}}
    \newcommand{\SpecialCharTok}[1]{\textcolor[rgb]{0.25,0.44,0.63}{{#1}}}
    \newcommand{\VerbatimStringTok}[1]{\textcolor[rgb]{0.25,0.44,0.63}{{#1}}}
    \newcommand{\SpecialStringTok}[1]{\textcolor[rgb]{0.73,0.40,0.53}{{#1}}}
    \newcommand{\ImportTok}[1]{{#1}}
    \newcommand{\DocumentationTok}[1]{\textcolor[rgb]{0.73,0.13,0.13}{\textit{{#1}}}}
    \newcommand{\AnnotationTok}[1]{\textcolor[rgb]{0.38,0.63,0.69}{\textbf{\textit{{#1}}}}}
    \newcommand{\CommentVarTok}[1]{\textcolor[rgb]{0.38,0.63,0.69}{\textbf{\textit{{#1}}}}}
    \newcommand{\VariableTok}[1]{\textcolor[rgb]{0.10,0.09,0.49}{{#1}}}
    \newcommand{\ControlFlowTok}[1]{\textcolor[rgb]{0.00,0.44,0.13}{\textbf{{#1}}}}
    \newcommand{\OperatorTok}[1]{\textcolor[rgb]{0.40,0.40,0.40}{{#1}}}
    \newcommand{\BuiltInTok}[1]{{#1}}
    \newcommand{\ExtensionTok}[1]{{#1}}
    \newcommand{\PreprocessorTok}[1]{\textcolor[rgb]{0.74,0.48,0.00}{{#1}}}
    \newcommand{\AttributeTok}[1]{\textcolor[rgb]{0.49,0.56,0.16}{{#1}}}
    \newcommand{\InformationTok}[1]{\textcolor[rgb]{0.38,0.63,0.69}{\textbf{\textit{{#1}}}}}
    \newcommand{\WarningTok}[1]{\textcolor[rgb]{0.38,0.63,0.69}{\textbf{\textit{{#1}}}}}
    
    
    % Define a nice break command that doesn't care if a line doesn't already
    % exist.
    \def\br{\hspace*{\fill} \\* }
    % Math Jax compatability definitions
    \def\gt{>}
    \def\lt{<}
    % Document parameters
    \title{Vector space\_subspace\_orthogonality\_projection}
    
    
    

    % Pygments definitions
    
\makeatletter
\def\PY@reset{\let\PY@it=\relax \let\PY@bf=\relax%
    \let\PY@ul=\relax \let\PY@tc=\relax%
    \let\PY@bc=\relax \let\PY@ff=\relax}
\def\PY@tok#1{\csname PY@tok@#1\endcsname}
\def\PY@toks#1+{\ifx\relax#1\empty\else%
    \PY@tok{#1}\expandafter\PY@toks\fi}
\def\PY@do#1{\PY@bc{\PY@tc{\PY@ul{%
    \PY@it{\PY@bf{\PY@ff{#1}}}}}}}
\def\PY#1#2{\PY@reset\PY@toks#1+\relax+\PY@do{#2}}

\expandafter\def\csname PY@tok@w\endcsname{\def\PY@tc##1{\textcolor[rgb]{0.73,0.73,0.73}{##1}}}
\expandafter\def\csname PY@tok@c\endcsname{\let\PY@it=\textit\def\PY@tc##1{\textcolor[rgb]{0.25,0.50,0.50}{##1}}}
\expandafter\def\csname PY@tok@cp\endcsname{\def\PY@tc##1{\textcolor[rgb]{0.74,0.48,0.00}{##1}}}
\expandafter\def\csname PY@tok@k\endcsname{\let\PY@bf=\textbf\def\PY@tc##1{\textcolor[rgb]{0.00,0.50,0.00}{##1}}}
\expandafter\def\csname PY@tok@kp\endcsname{\def\PY@tc##1{\textcolor[rgb]{0.00,0.50,0.00}{##1}}}
\expandafter\def\csname PY@tok@kt\endcsname{\def\PY@tc##1{\textcolor[rgb]{0.69,0.00,0.25}{##1}}}
\expandafter\def\csname PY@tok@o\endcsname{\def\PY@tc##1{\textcolor[rgb]{0.40,0.40,0.40}{##1}}}
\expandafter\def\csname PY@tok@ow\endcsname{\let\PY@bf=\textbf\def\PY@tc##1{\textcolor[rgb]{0.67,0.13,1.00}{##1}}}
\expandafter\def\csname PY@tok@nb\endcsname{\def\PY@tc##1{\textcolor[rgb]{0.00,0.50,0.00}{##1}}}
\expandafter\def\csname PY@tok@nf\endcsname{\def\PY@tc##1{\textcolor[rgb]{0.00,0.00,1.00}{##1}}}
\expandafter\def\csname PY@tok@nc\endcsname{\let\PY@bf=\textbf\def\PY@tc##1{\textcolor[rgb]{0.00,0.00,1.00}{##1}}}
\expandafter\def\csname PY@tok@nn\endcsname{\let\PY@bf=\textbf\def\PY@tc##1{\textcolor[rgb]{0.00,0.00,1.00}{##1}}}
\expandafter\def\csname PY@tok@ne\endcsname{\let\PY@bf=\textbf\def\PY@tc##1{\textcolor[rgb]{0.82,0.25,0.23}{##1}}}
\expandafter\def\csname PY@tok@nv\endcsname{\def\PY@tc##1{\textcolor[rgb]{0.10,0.09,0.49}{##1}}}
\expandafter\def\csname PY@tok@no\endcsname{\def\PY@tc##1{\textcolor[rgb]{0.53,0.00,0.00}{##1}}}
\expandafter\def\csname PY@tok@nl\endcsname{\def\PY@tc##1{\textcolor[rgb]{0.63,0.63,0.00}{##1}}}
\expandafter\def\csname PY@tok@ni\endcsname{\let\PY@bf=\textbf\def\PY@tc##1{\textcolor[rgb]{0.60,0.60,0.60}{##1}}}
\expandafter\def\csname PY@tok@na\endcsname{\def\PY@tc##1{\textcolor[rgb]{0.49,0.56,0.16}{##1}}}
\expandafter\def\csname PY@tok@nt\endcsname{\let\PY@bf=\textbf\def\PY@tc##1{\textcolor[rgb]{0.00,0.50,0.00}{##1}}}
\expandafter\def\csname PY@tok@nd\endcsname{\def\PY@tc##1{\textcolor[rgb]{0.67,0.13,1.00}{##1}}}
\expandafter\def\csname PY@tok@s\endcsname{\def\PY@tc##1{\textcolor[rgb]{0.73,0.13,0.13}{##1}}}
\expandafter\def\csname PY@tok@sd\endcsname{\let\PY@it=\textit\def\PY@tc##1{\textcolor[rgb]{0.73,0.13,0.13}{##1}}}
\expandafter\def\csname PY@tok@si\endcsname{\let\PY@bf=\textbf\def\PY@tc##1{\textcolor[rgb]{0.73,0.40,0.53}{##1}}}
\expandafter\def\csname PY@tok@se\endcsname{\let\PY@bf=\textbf\def\PY@tc##1{\textcolor[rgb]{0.73,0.40,0.13}{##1}}}
\expandafter\def\csname PY@tok@sr\endcsname{\def\PY@tc##1{\textcolor[rgb]{0.73,0.40,0.53}{##1}}}
\expandafter\def\csname PY@tok@ss\endcsname{\def\PY@tc##1{\textcolor[rgb]{0.10,0.09,0.49}{##1}}}
\expandafter\def\csname PY@tok@sx\endcsname{\def\PY@tc##1{\textcolor[rgb]{0.00,0.50,0.00}{##1}}}
\expandafter\def\csname PY@tok@m\endcsname{\def\PY@tc##1{\textcolor[rgb]{0.40,0.40,0.40}{##1}}}
\expandafter\def\csname PY@tok@gh\endcsname{\let\PY@bf=\textbf\def\PY@tc##1{\textcolor[rgb]{0.00,0.00,0.50}{##1}}}
\expandafter\def\csname PY@tok@gu\endcsname{\let\PY@bf=\textbf\def\PY@tc##1{\textcolor[rgb]{0.50,0.00,0.50}{##1}}}
\expandafter\def\csname PY@tok@gd\endcsname{\def\PY@tc##1{\textcolor[rgb]{0.63,0.00,0.00}{##1}}}
\expandafter\def\csname PY@tok@gi\endcsname{\def\PY@tc##1{\textcolor[rgb]{0.00,0.63,0.00}{##1}}}
\expandafter\def\csname PY@tok@gr\endcsname{\def\PY@tc##1{\textcolor[rgb]{1.00,0.00,0.00}{##1}}}
\expandafter\def\csname PY@tok@ge\endcsname{\let\PY@it=\textit}
\expandafter\def\csname PY@tok@gs\endcsname{\let\PY@bf=\textbf}
\expandafter\def\csname PY@tok@gp\endcsname{\let\PY@bf=\textbf\def\PY@tc##1{\textcolor[rgb]{0.00,0.00,0.50}{##1}}}
\expandafter\def\csname PY@tok@go\endcsname{\def\PY@tc##1{\textcolor[rgb]{0.53,0.53,0.53}{##1}}}
\expandafter\def\csname PY@tok@gt\endcsname{\def\PY@tc##1{\textcolor[rgb]{0.00,0.27,0.87}{##1}}}
\expandafter\def\csname PY@tok@err\endcsname{\def\PY@bc##1{\setlength{\fboxsep}{0pt}\fcolorbox[rgb]{1.00,0.00,0.00}{1,1,1}{\strut ##1}}}
\expandafter\def\csname PY@tok@kc\endcsname{\let\PY@bf=\textbf\def\PY@tc##1{\textcolor[rgb]{0.00,0.50,0.00}{##1}}}
\expandafter\def\csname PY@tok@kd\endcsname{\let\PY@bf=\textbf\def\PY@tc##1{\textcolor[rgb]{0.00,0.50,0.00}{##1}}}
\expandafter\def\csname PY@tok@kn\endcsname{\let\PY@bf=\textbf\def\PY@tc##1{\textcolor[rgb]{0.00,0.50,0.00}{##1}}}
\expandafter\def\csname PY@tok@kr\endcsname{\let\PY@bf=\textbf\def\PY@tc##1{\textcolor[rgb]{0.00,0.50,0.00}{##1}}}
\expandafter\def\csname PY@tok@bp\endcsname{\def\PY@tc##1{\textcolor[rgb]{0.00,0.50,0.00}{##1}}}
\expandafter\def\csname PY@tok@fm\endcsname{\def\PY@tc##1{\textcolor[rgb]{0.00,0.00,1.00}{##1}}}
\expandafter\def\csname PY@tok@vc\endcsname{\def\PY@tc##1{\textcolor[rgb]{0.10,0.09,0.49}{##1}}}
\expandafter\def\csname PY@tok@vg\endcsname{\def\PY@tc##1{\textcolor[rgb]{0.10,0.09,0.49}{##1}}}
\expandafter\def\csname PY@tok@vi\endcsname{\def\PY@tc##1{\textcolor[rgb]{0.10,0.09,0.49}{##1}}}
\expandafter\def\csname PY@tok@vm\endcsname{\def\PY@tc##1{\textcolor[rgb]{0.10,0.09,0.49}{##1}}}
\expandafter\def\csname PY@tok@sa\endcsname{\def\PY@tc##1{\textcolor[rgb]{0.73,0.13,0.13}{##1}}}
\expandafter\def\csname PY@tok@sb\endcsname{\def\PY@tc##1{\textcolor[rgb]{0.73,0.13,0.13}{##1}}}
\expandafter\def\csname PY@tok@sc\endcsname{\def\PY@tc##1{\textcolor[rgb]{0.73,0.13,0.13}{##1}}}
\expandafter\def\csname PY@tok@dl\endcsname{\def\PY@tc##1{\textcolor[rgb]{0.73,0.13,0.13}{##1}}}
\expandafter\def\csname PY@tok@s2\endcsname{\def\PY@tc##1{\textcolor[rgb]{0.73,0.13,0.13}{##1}}}
\expandafter\def\csname PY@tok@sh\endcsname{\def\PY@tc##1{\textcolor[rgb]{0.73,0.13,0.13}{##1}}}
\expandafter\def\csname PY@tok@s1\endcsname{\def\PY@tc##1{\textcolor[rgb]{0.73,0.13,0.13}{##1}}}
\expandafter\def\csname PY@tok@mb\endcsname{\def\PY@tc##1{\textcolor[rgb]{0.40,0.40,0.40}{##1}}}
\expandafter\def\csname PY@tok@mf\endcsname{\def\PY@tc##1{\textcolor[rgb]{0.40,0.40,0.40}{##1}}}
\expandafter\def\csname PY@tok@mh\endcsname{\def\PY@tc##1{\textcolor[rgb]{0.40,0.40,0.40}{##1}}}
\expandafter\def\csname PY@tok@mi\endcsname{\def\PY@tc##1{\textcolor[rgb]{0.40,0.40,0.40}{##1}}}
\expandafter\def\csname PY@tok@il\endcsname{\def\PY@tc##1{\textcolor[rgb]{0.40,0.40,0.40}{##1}}}
\expandafter\def\csname PY@tok@mo\endcsname{\def\PY@tc##1{\textcolor[rgb]{0.40,0.40,0.40}{##1}}}
\expandafter\def\csname PY@tok@ch\endcsname{\let\PY@it=\textit\def\PY@tc##1{\textcolor[rgb]{0.25,0.50,0.50}{##1}}}
\expandafter\def\csname PY@tok@cm\endcsname{\let\PY@it=\textit\def\PY@tc##1{\textcolor[rgb]{0.25,0.50,0.50}{##1}}}
\expandafter\def\csname PY@tok@cpf\endcsname{\let\PY@it=\textit\def\PY@tc##1{\textcolor[rgb]{0.25,0.50,0.50}{##1}}}
\expandafter\def\csname PY@tok@c1\endcsname{\let\PY@it=\textit\def\PY@tc##1{\textcolor[rgb]{0.25,0.50,0.50}{##1}}}
\expandafter\def\csname PY@tok@cs\endcsname{\let\PY@it=\textit\def\PY@tc##1{\textcolor[rgb]{0.25,0.50,0.50}{##1}}}

\def\PYZbs{\char`\\}
\def\PYZus{\char`\_}
\def\PYZob{\char`\{}
\def\PYZcb{\char`\}}
\def\PYZca{\char`\^}
\def\PYZam{\char`\&}
\def\PYZlt{\char`\<}
\def\PYZgt{\char`\>}
\def\PYZsh{\char`\#}
\def\PYZpc{\char`\%}
\def\PYZdl{\char`\$}
\def\PYZhy{\char`\-}
\def\PYZsq{\char`\'}
\def\PYZdq{\char`\"}
\def\PYZti{\char`\~}
% for compatibility with earlier versions
\def\PYZat{@}
\def\PYZlb{[}
\def\PYZrb{]}
\makeatother


    % Exact colors from NB
    \definecolor{incolor}{rgb}{0.0, 0.0, 0.5}
    \definecolor{outcolor}{rgb}{0.545, 0.0, 0.0}



    
    % Prevent overflowing lines due to hard-to-break entities
    \sloppy 
    % Setup hyperref package
    \hypersetup{
      breaklinks=true,  % so long urls are correctly broken across lines
      colorlinks=true,
      urlcolor=urlcolor,
      linkcolor=linkcolor,
      citecolor=citecolor,
      }
    % Slightly bigger margins than the latex defaults
    
    \geometry{verbose,tmargin=1in,bmargin=1in,lmargin=1in,rmargin=1in}
    
    

    \begin{document}
    
    
    \maketitle
    
    

    
    \section{Vector space, subspaces}\label{vector-space-subspaces}

\begin{itemize}
\tightlist
\item
  A vector space is a set that is closed under finite vector addition
  and scalar multiplication.
\end{itemize}

\subsection{Four subspaces}\label{four-subspaces}

For \(m \times n\) matrix \(A\),\(rank(A)=r\), we have:

\begin{itemize}
\item
  row space \(C(A^T) \in \mathbb{R}^n, dim C(A^T)=r\)
\item
  null space \(N(A) \in \mathbb{R}^n, dim N(A)=n-r\)
\item
  column \(C(A) \in \mathbb{R}^m, dim C(A)=r\)
\item
  left null \(N(A^T) \in \mathbb{R}^m, dim N(A^T)=m-r\). The reason why
  it is called left null space:
  \(A^Ty=0 \rightarrow (A^Ty)^T=0^T\rightarrow y^TA=0^T\)
\item
  For \(A_{mn}\), column space is the subspace of \(R^{m}\) and null
  space is the subspace of \(R^{n}\).
\item
  A null space example. For a particular matrix
  \(A_{mn} \equiv A_{24}\), we obtain a system of two equations but four
  variables (more variables than equations). For example
  \(x+2y-z+s = 2\) and \(2x+46-2z+2s = 4\). We have two free variables z
  and s and can be set with two independent values (z,s) = (0,1) or
  (1,0). Whenever, we have free variables, we have non-zero dimension
  null space. This happens when \(m>n\).
\item
  We have now two ways of creating a subspace. (1) First we use the
  linear combination of columns of a matrix and obtain the so called
  column space \(C(A)\). Second we can use the solutions to \(Ax = 0\)
  for form a null space, which is a subspace. The key point is that the
  two ways both guarantee that the zero vector is included. For row
  space and left null space, it is similar.
\item
  Note we cannot use the solution of \(Ax = b, b\neq 0\) to form a
  subspace. (1) In this case \(x=0\) is not necessarily the solution to
  the equation. However, we require any vector subspace must have zero
  vector. (2) The solution to \(Ax = b\) is \(x=x_{n}+x_{p} = x_{p}\),
  which is given by a null subspace vector shifted by a particular
  vector, and thus no longer is a subspace (not going through zero
  vector anymore).
\item
  \begin{enumerate}
  \def\labelenumi{(\roman{enumi})}
  \tightlist
  \item
    row operations (such as eliminations) do not change the row space
    and null space (see the link below); (ii) column operations do not
    change the column space (and left-null space?). But how to
    understand row operations does not affect null space.
    https://math.stackexchange.com/questions/108041/linear-algebra-preserving-the-null-space.\\
  \end{enumerate}
\item
  The sum dimensions \(C(A)\) and \(N(A)\) is \(n\) for the
  \(m\times n\) matrix.
\item
  A vectors such as \(v\in\mathbb{R}^4\), called a 4-dimensional vector.
  However, we cannot call a matrix contains such vectors as a
  four-dimensional space. As studied earlier, the dimension of four
  subspaces for this matrix can be very different.\\
\item
  When plotting two vectors \(a\) and \(b\) on a paper, NEVER think that
  they must be 2D vectors (\(a or b \in \mathbb{R}^2\)). In fact, they
  might be true that \(a or b \in \mathbb{R}^4\), etc.
\item
  A very important relation: \(dim C(A) = dim C(A^T)\). This means that
  the rank or dimension of column space is equal to that of row space.
  Be careful, never judge whether a matrix is a basis or invertible only
  from relations of columns.
\item
  Full rank matrix is not necessarily a square matrix. We need separate
  column full rank, row full rank, and both row and column full rank.
\end{itemize}

\subsection{\texorpdfstring{Solution structure of
\(Ax = b\)}{Solution structure of Ax = b}}\label{solution-structure-of-ax-b}

\begin{itemize}
\tightlist
\item
  \(Ax = b\) has solution if and only if \(b\in C(A)\)
\end{itemize}

\[\begin{array}{c|c|c|c}r=m=n&r=n\lt m&r=m\lt n&r\lt m,r\lt n\\R=I&R=\begin{bmatrix}I\\0\end{bmatrix}&R=\begin{bmatrix}I&F\end{bmatrix}&R=\begin{bmatrix}I&F\\0&0\end{bmatrix}\\1\ solution&0\ or\ 1\ solution&\infty\ solution&0\ or\ \infty\ solution\end{array}\]

\begin{itemize}
\tightlist
\item
  There are two major groups: (1) Both column and row full rank. (2)
  Other case which include 2-a: full-column rank, 2-b: full-row rank,
  2-c: both row and column are not full rank. All these cases can be
  understood with geometric interpretation.
\item
  1-a. \(r=m=n\): \(Ax = 0\) only have a solution \(x=0\). Or we can
  understand as the number of free variable is zero and thus null space
  is only \(x=0\). So the complete solution is
  \(x=x_{n}+x_{p} = x_{p}\). In other words, we have solution and only
  one particular solution. I focused on only this type of solution for
  many years. The unique solution is \(x=A^{-1}b\). The reason why we
  must have a unique solution is that \(b\) and column space are with
  the same dimension, and thus \(b\) can sure be expanded in the column
  space.
\item
  2-a. Only column full rank (memorize as tall matrix). Imagine a
  \(4\times2\) matrix. Full column rank indicates two independent column
  vectors in \(\mathbb{R}^4\) and thus \(Ax=0\) has only \(x=0\)
  solution. If \(Ax=b\) has solution, then it requires that the \(b\)
  must lie in the \(C(A)\). If this is the case, then we have ONE
  solution. However, it is possible that two column vectors of \(A\)
  cannot linearly combine to any four-dimensional vector \(b\). In this
  later case, there is no solution.
\item
  2-b. Only row full rank (memorize as fat matrix). Imagine a
  \(2\times4\) matrix. We thus know \(C(A)\in \mathbb{R}^2\). As shown
  in lecture 10, \(dim C(A) = dim C(A^T)\), so if there are two
  independent rows, then there must be two independent columns.
  Therefore the four columns (two of them are independent) in \(A\) can
  linearly combine into any \(b\) vector in \(\mathbb{R}^2\). In other
  words, we can always find at least one particular solution \(x_p\) to
  \(Ax = b\). Now we turn to the solution to \(Ax = 0\), i.e., the
  \(N(A)\). The null space dimension is 4-2 = 2 and thus we have two
  free variables, or have two independent vector in null space. We can
  obtain the \(x_n\), which is linear combination of two
  \(\mathbb{R}^4\) vectors. Then the final solution will be
  \(x = x_n+x_p\). The key is this fat matrix case, in 2-dimensional
  space, there are at least two independent column vectors, and thus we
  can expand any vector \(b\), and thus we have at least one particular
  solution. By the way, the \(N(A)\) is a combination of two independent
  vectors, and thus is a plane in \(\mathbb{R}^4\) space. Finally, from
  above, the non-zero \(N(A)\) and the existing \(x_p\) give infinite
  number of solutions.
\item
  2-c If column is not full rank, it is like the case of 2-b and we may
  have a particular solution if \(b \in C(A)\). Otherwise no nonzero
  \(x_p\). If row is not full rank, it is like the case of 2-a, we have
  nonzero \(N(A)\) and thus may have infinity number of solution. So in
  this case, there might be no solution or \(\infty\) solutions.
\end{itemize}

    \section{Orthogonality}\label{orthogonality}

\subsection{Orthogonality of four
subspaces}\label{orthogonality-of-four-subspaces}

For \(m \times n\) matrix \(A\) with rank \(r\), the row space
(\(dim C(A^T)=r\)) and null space (\(dim N(A)=n-r\)) are orthogonal to
each other and both belong to \(\mathbb{R}^n\). The column space
(\(dim C(A)=r\)) and left null space (\(dim N(A^T)\)=m-r) are also
orthogonal to each other and belong to \(\mathbb{R}^m\).

A best imagination to understand the orthogonality of the four
subspaces. (A) in the matrix form of \(Ax = 0\), note the rows of \(A\)
(horizontal direction) are perpendicular to the \(x\) vector (vertical
direction). Thus the \(N(A)\) where \(x\) resides is orthogonal to the
row space of \(A\). (B)In the matrix form of \(x^TA = 0\), the \(x\) is
perpendicular to the column vector. So left null space is equal to
column space of \(A\).

When subspace \(S\) and subspace \(T\) are orthogonal, then any vector
in \(S\) is orthogonal to any vectors in \(T\). Two walls cannot be two
orthogonal spaces.

Example: For \(A=\begin{bmatrix}1&2&5\\2&4&10\end{bmatrix}\), we have
\(m=2, n=3, rank(A)=1, dim N(A)=2\). From
\(Ax=\begin{bmatrix}1&2&5\\2&4&10\end{bmatrix}\begin{bmatrix}x_1\\x_2\\x_3\end{bmatrix}=\begin{bmatrix}0\\0\end{bmatrix}\),
we obtain a basis of null space:
\(x_1=\begin{bmatrix}-2\\1\\0\end{bmatrix}\quad x_2=\begin{bmatrix}-5\\0\\1\end{bmatrix}\).
The basis in row space is \(r=\begin{bmatrix}1\\2\\5\end{bmatrix}\). The
null space is orthogonal to the row space. In this example, row space is
the normal vector to the null space.

Row space and null space are orthogonal complement in \(n\) dimensional
space, i.e., null space contains all the vectors that are perpendicular
to row space. Similarly, column space and left null space are orthogonal
complement in \(m\) dimensional space.

    \subsection{Projection to subspaces}\label{projection-to-subspaces}

\subsubsection{General picture}\label{general-picture}

\begin{itemize}
\tightlist
\item
  First have a clear picture on how column space \(C(A)\bot N(A^T)\).
  Review the imagination picture earlier.
\item
  We want to solve \(Ax = b\), where \(A\) is a tall matrix. We do not
  have a solution because \(C(A)\) cannot expand vector \(b\). For tall
  matrix, \(b\) might be a four-dimensional vector but \(C(A)\) has a
  dimension of three. Note even for fat matrix, \(C(A)\) might not
  expand vector \(b\). \textbf{So it is not just for tall matrix that
  \(Ax = b\) might have no solution.}
\item
  Because there are no solution in \(C(A)\), we want an approximate
  solution \(\hat{x}\). From the projection picture, \(b\)'s projection
  to \(C(A)\) is \(p = A\hat{x}\).
\item
  The error or difference between \(b\) and \(p\) is
  \(e = b-p = b - A\hat{x}\).
\item
  Also from the projection picture, \(e\) is normal to all the columns
  of \(A\). Thus we have \(a_1^T(b-p) = a_1^Te = 0\),
  \(a_2^T(b-p) = a_2^Te = 0\)... These can be written as
  \(A^T e = A^T(b-A\hat{x}) = 0\), or \(e\) is in the \(N(A^T)\), the
  left null space of \(A\). From this we can obtain
  \(\hat{x} = (A^TA)^{-1}A^Tb\), which is the approximate solution of
  \(Ax = b\).
\item
  The above picture of \(e\) perpendicular to \(C(A)\) is extremely
  important. All the ideas of introducing normal matrix, normal
  equation, etc. are all from this geometric picture.
\end{itemize}

\subsubsection{2D Projection}\label{d-projection}

In projection, vector \(e\) is the error between two vectors\(b, p\),
\(e=b-p, e \bot p\). Projection \(p\) is on \(a\) and \(p=ax\). Thus we
have \(a^Te=a^T(b-p)=a^T(b-ax)=0\). We have the very important
equation,\$ a\^{}T(b-xa)=0, xa\textsuperscript{Ta=a}Tb,
x=\frac{a^Tb}{a^Ta}, p=a\frac{a^Tb}{a^Ta}\$.

\(P=\frac{aa^T}{a^Ta}\). If \(a\) is \(n\) dimensional column vector,
then \(P\) is a \(n \times n\) matrix. The column space of projection
matrix \(C(P)\) is a line going through \(a\) with \(rank(P)=1\). \(a\)
is the basis of this matrix.

Projection is a symmetric matrix \(P=P^T\) and thus can always be
diagonalizable. Also we have \(P^2=P\).

    \subsubsection{3D projection}\label{d-projection}

The projection matrix in 3D can be derived with the similar way in 2D
case (see details in course notes). However, here we will use a simple
way to understand. The 2D case \(P=\frac{aa^T}{a^Ta}\) becomes
\(P=aa^T\) when \(a\) is a unit vector. Then \(Px = aa^Tx = a_ia\),
where \(a_i\) is the projection of vector \(x\) on the vector \(a\).
This is exactly same as the projection operator in physics which usually
described by Dirac notation.\\
Extending to 3D case \(P=AA^T\) gives \(Px = AA^Tx\). Each row of
\(A^T\) dot multiply with vector \(x\) give the value, which is the
projection of \(x\) on this row vector of \(A^T\) or column vector of
\(A\). This gives all the projections of \(x\) on all the column vectors
of \(A\). The projections is denoted by a vector \(y\) and thus
\(Px = Ay\). From the column picture of matrix multiplication, \(Ay\) is
just expand with the columns of \(A\). In other words, \(Px\) projects
\(x\) into the column space of \(A\). If the column vectors of \(A\) is
not unit vectors, then \(P = A(A^TA)^{-1}A^T\).\\
\textbf{Important conclusion}: * If \(A\) contains \textbf{unit column
vectors}, then \(AA^Tx\) projects \(x\) onto the column space of \(A\).
* If \(A\) contains \textbf{unit row vectors } (required?), then
\(A^TAx\) projects \(x\) onto the row space of \(A\). So \(Ax\) may have
interpreted as: (1) column expansion with coefficients in \(x\). (2).
\(Ax = y\), then \(y\) is the projections of \(x\) on the row space (the
row vectors need to be unit vector).

    \subsubsection{A summary to memorize}\label{a-summary-to-memorize}

\begin{itemize}
\tightlist
\item
  \(Ax = b \Rightarrow A\hat{x} = p \Rightarrow A\hat{x} = Pb = A(A^TA)^{-1}A^Tb \Rightarrow \hat{x} = (A^TA)^{-1}A^Tb\).
  From the pseudo-inverse part, we know that the projection matrix here
  projecting \(\mathbb{R}^m\) to column space \(C(A)\). The projection
  matrix can be written as \(A\) multiplied by left inverse matrix
  \(A_{left}^{-1} = (A^TA)^{-1}A^T\). Also from \(Ax = b\) we know that
  the linear transformation \(A\) transform a vector \(x\) in row space
  \(C(A^T)\) to column space \(C(A)\).
\item
  \(A\) left multiplying left inverse gives projection matrix which
  projects \(\mathbb{R}^m\) to \(C(A)\). If we right multiply A to right
  inverse, then it will give a projection matrix projecting
  \(\mathbb{R}^n\) to row space \(C(A^T)\).
  \(A_{right}^{-1} = A^T(AA^T)^{-1}\). Thus the projecting matrix is
  \(p = A^T(AA^T)^{-1}A\).
\end{itemize}

    \subsubsection{Minimum Least Square}\label{minimum-least-square}

The minimum least square method is exactly mapped to the problem of
solving \(Ax = b\). So I can understand everything here with the
conclusions arrived earlier. In the following results, we need find the
clear geometric significance of terms such as \(p, e, \hat{x}\). Also
what is \(C(A), N(A^T)\), etc.

We will find a line \(b=C+Dt\) that has minimum deviation from three
points \((1, 1), (2, 2), (3, 2)\). From this we have the following
equations \$

\begin{cases}
C+D&=1 \\
C+2D&=2 \\
C+3D&=2 \\
\end{cases}

\$. The matrix form is
\(\begin{bmatrix}1&1 \\1&2 \\1&3\\\end{bmatrix}\begin{bmatrix}C\\D\\\end{bmatrix}=\begin{bmatrix}1\\2\\2\\\end{bmatrix}\).
That is \(Ax=b\). Obviously, there is no solution to the system of
equations. However, there is solution to \(A^TA\hat x=A^Tb\).
Multiplying \(A^T\) in both sides gives \(A^TA\hat x=A^Tb\), which is
the fundamental equation for minimum least square.

    Now we are finding the solutions
\(\hat x=\begin{bmatrix}\hat C\\ \hat D\end{bmatrix}\)与\(p=\begin{bmatrix}p_1\\p_2\\p_3\end{bmatrix}\).

\[
A^TA\hat x=A^Tb\\
A^TA=
\begin{bmatrix}3&6\\6&14\end{bmatrix}\qquad
A^Tb=
\begin{bmatrix}5\\11\end{bmatrix}\\
\begin{bmatrix}3&6\\6&14\end{bmatrix}
\begin{bmatrix}\hat C\\\hat D\end{bmatrix}=
\begin{bmatrix}5\\11\end{bmatrix}\\
\]

Converting to equations gives
\(\begin{cases}3\hat C+16\hat D&=5\\6\hat C+14\hat D&=11\\\end{cases}\),
which are also called normal equations. The solutions are
\(\hat C=\frac{2}{3}, \hat D=\frac{1}{2}\), corresponding to the 'best
line' \(y=\frac{2}{3}+\frac{1}{2}t\). Plugging into the original
equations gives \(p_1=\frac{7}{6}, p_2=\frac{5}{3}, p_3=\frac{13}{6}\).
That is, \(e_1=-\frac{1}{6}, e_2=\frac{1}{3}, e_3=-\frac{1}{6}\).

Thus we have
\(p=\begin{bmatrix}\frac{7}{6}\\\frac{5}{3}\\\frac{13}{6}\end{bmatrix}, e=\begin{bmatrix}-\frac{1}{6}\\\frac{1}{3}\\-\frac{1}{6}\end{bmatrix}\).
Obviously, \(b=p+e\) and \(p\cdot e=0\), i.e., \(p\bot e\). The error
vector \(e\) is not only perpendicular to \(p\) but also to the whole
column space. For example,
\(\begin{bmatrix}1\\1\\1\end{bmatrix}, \begin{bmatrix}1\\2\\3\end{bmatrix}\).

    \subsubsection{Why projection is
introduced?}\label{why-projection-is-introduced}

\begin{itemize}
\tightlist
\item
  Projection is strongly related to orthogonality.
\item
  Projection is also strongly related to optimization, i.e. minimization
  or maximization. From the geometrical picture, we know a that a
  projection is automatically the optimal solution is the space
  projected into. The key reason is as follows: \(e\) is the error
  between two vectors, and \(e\) is minimized when vertically
  (orthogonally) projected to another space.
\item
  To let the normal equations solvable, we need the \(A\) is full column
  rank, as shown below.
\end{itemize}

If the columns of matrix \(A\) are linearly independent, then \(A^TA\)
is invertible.

First assume \(A^TAx=0\), and then multiplying \(x^T\) in both sides
gives \(x^TA^TAx=0\), i.e., \((Ax)^T(Ax)=0\). Thus \(Ax=0\). Because
columns of \(A\) are independent, thus the null space of \(A\) only has
zero vector.

    \subsection{Orthonormal matrix and
Gram-Schmidt}\label{orthonormal-matrix-and-gram-schmidt}

orthonormal:\(q_i^Tq_j=\begin{cases}0\quad i\neq j\\1\quad i=j\end{cases}\)\\
\(Q=\Bigg[q_1 q_2 \cdots q_n\Bigg]\)\\
\(Q^TQ=I\) \(Q^T=Q^{-1}\)

Advantage of orthonormal matrix:\\
For ordinary matrix, the projection matrix to its column space is
complicated. For orthonormal matrix, if we want to project vector \(b\)
to the column space of matrix \(Q\), then we have a much simpler
projection matrix, \(P=Q(Q^TQ)^{-1}Q^T = QQ^T\). Therefore, when columns
of a matrix is orthonormal, then \(QQ^T\) is the projection matrix. In
special case of square matrix, then if its columns are orthonormal, then
its column space is the whole vector space, and the projection matrix to
this whole space is identity matrix, i.e., \(QQ^T=I\).

Steps of Gram-Schmidt:

For two linearly independent vectors \(a, b\), first transfer them into
two perpendicular vectors \(A, B\), and then normalize them with
\(q_1=\frac{A}{\left\|A\right\|}, q_2=\frac{B}{\left\|B\right\|}\):

\begin{itemize}
\tightlist
\item
  Let \(a=A\)\\
\item
  Projecting \(b\) onto the direction of normal vector of \(A\) gives
  vector \(B\). This is just the error vector introduced earlier
  \(e=b-p\). That is \(B=b-\frac{A^Tb}{A^TA}A\). Now verify whether
  \(A\bot B\) is true.
  \(A^TB=A^Tb-A^T\frac{A^Tb}{A^TA}A=A^Tb-\frac{A^TA}{A^TA}A^Tb=0\).
  (\(\frac{A^Tb}{A^TA}A\) is \(A\hat x=p\)).
\end{itemize}

For three independent vectors \(a, b, c\), we need first find their
corresponding orthogonal vectors \(A, B, C\) and then normalize them
with
\(q_1=\frac{A}{\left\|A\right\|}, q_2=\frac{B}{\left\|B\right\|}, q_3=\frac{C}{\left\|C\right\|}\):

\begin{itemize}
\tightlist
\item
  We have shown how to obtain the first two orthogonal vectors. Now we
  need find the third vector which is orthogonal to both \(A, B\).\\
\item
  Following the similar approach, we first calculate the projection of
  \(c\) on \(A, B\), and then subtract this projection from \(c\).
  \(C=c-\frac{A^Tc}{A^TA}A-\frac{B^Tc}{B^TB}B\).
\end{itemize}

    \subsection{Properties of orthogonal matrix (unitary
matrix)}\label{properties-of-orthogonal-matrix-unitary-matrix}

\begin{itemize}
\tightlist
\item
  It preserves the norm of a vector. An example is rotational matrix.
  This is easy to prove:\\
  \(\left\|Ux\right\|^2 = (Ux)^T(Ux) = x^TU^TUx = x^Tx = \left\|x\right\|^2\)
\end{itemize}


    % Add a bibliography block to the postdoc
    
    
    
    \end{document}
